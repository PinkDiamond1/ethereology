\documentclass[a4paper,oneside]{article}
\usepackage{graphicx}
\usepackage[utf8]{inputenc}
\usepackage{ragged2e}
\usepackage[english]{babel}
%\usepackage{draftwatermark}
\usepackage{lipsum}
\usepackage[scaled=.92]{montserrat}
\usepackage{color,calc}
\usepackage[T1]{fontenc}
\usepackage{montserrat}
\usepackage[edges]{forest}
\usepackage{dirtytalk}
\usepackage{listings}
\usepackage{color}
\usepackage{multicol}
\usepackage{setspace}
\usepackage[normalem]{ulem}
\usepackage{geometry}


\title{Ethereology: Observational Ethereum}
\author{Micah Dameron}
\renewcommand{\baselinestretch}{1.15}


\begin{document}
\maketitle

\begin{abstract}
Ethereum is poised to change the way the world operates in many fundamental ways. It increases certainty, both for individuals and society, by several orders of magnitude. It gives us a single source of truth to rely on that isn't nearly as subject to interpretation as the institutions and models we're used to. This will revolutionize everything the hand of humanity touches. To make the implications of this system more explicit, a platform is proposed that can classify the objects and processes of the Ethereum blockchain directly, as they relate to certain queries about existential data.

\end{abstract}

\hfill

\begin{multicols}{2}

\section{Artifact Explorer}
With every new block, Ethereum creates permanent artifacts. These are sociological and historical details that will outlast our own records of societies which came before us.

\section{Data Processing Language}
Create and generate custom databases using the suite of search tools.

\section{Philosophical Research}
Research papers integrating data objects from Ethereum into everyday experience as real existential entities.

\section{Iterative, Evolving Specification}
A core, fundamental standard that is authoritative by nature of its integration of all primary elements of Ethereum, and all secondary elements (or as many as possible.)


\end{multicols}

\end{document}
