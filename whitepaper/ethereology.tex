\documentclass[a4paper,oneside]{article}
\usepackage{graphicx}
\usepackage[utf8]{inputenc}
\usepackage{ragged2e}
\usepackage[english]{babel}
%\usepackage{draftwatermark}
\usepackage{lipsum}
\usepackage[scaled=.92]{montserrat}
\usepackage{color,calc}
\usepackage[T1]{fontenc}
\usepackage{montserrat}
\usepackage[edges]{forest}
\usepackage{dirtytalk}
\usepackage{listings}
\usepackage{color}
\usepackage{multicol}
\usepackage{setspace}
\usepackage[normalem]{ulem}
\usepackage{geometry}



\title{Ethereology: The Science of Ethereum}
\author{Micah Dameron}
\renewcommand{\baselinestretch}{1.15}


\begin{document}
\maketitle

\begin{abstract}
Ethereum is poised to change the way the world operates in many fundamental ways. It increases certainty, both for individuals and society, by several orders of magnitude. It gives us a single source of truth to rely on that isn't nearly as subject to interpretation as the institutions and models we're used to. 
\end{abstract}

\begin{multicols}{2}

\section{Facilitating Interaction}
The key word is enablement. Ethereum  doesn't necessarily provide humanity with a new way of operating. Rather it forcefully interjects a layer of truth between interactions we already have in our old ways of operating. The goal is not a trustless society, but rather one where trust isn't wasted on institutions that control, temporarily or permanently, offerings to certain life services.

This change will be so in-depth, and so profound, that it will fundamentally and permanently change the primary operation of institutions and individuals across society. So far-reaching and comprehensive will be these changes that the world as we know it today will barely resemble itself in 100 years. We are on the brink of societal and cultural change that is going to make the Industrial Revolution and the Information Revolution look like peanuts. 

Until now humanity has made its living through conflict. Conflicts on the inside, and conflicts on the outside. Conflicts will still exist, of course. But until now our conflicts have held us back significantly, because of their power to destroy. Besides the natural imperative to reproduce and propagate oneself and one's life, there are no forces fundamentally driving the survival of the human race as a whole (let alone that of individual humans.) 

Sure we have our teamwork and our ability to communicate, but that, even under the best of circumstances, these can be so flimsy as to repeatedly fail at the precise moments when the situation calls for them. This is why leadership is always a valued asset. Anyone with the tenacity to bring together the tenuous connections that bind most people into an organized whole seems to be intrinsically worth following. 

None of these positive personality attributes are minimized or neglected by the operation of inherently trustworthy protocols like Ethereum. On the contrary, the actions of the trustworthy protocols allow us to automate commercial and public transactions so that information which is normally garnered trust needs can remain out of commercial transactions where it belongs.

\paragraph{Trust} is a privilege to receive, and one's own right to give or take. Trust fosters close human relationships, and most of the things that make life worth living. Trust is an inherent property in all positive human relationships. Trust is abused when one party gathers information from another and doesn't use that information in their best interests. This is what companies do today.

This is not to say that companies, or corporations are evil per se. They may or may not be. It depends on how they abuse the trust, and use the trust that is given to them. If we step back a few centuries to older societal structures, you'll find that there were kings and nobles, as well as other institutions, that abused people's trust. A thousand years before that, conquesting and "taking" cities was considered a victory. In reality it was an abuse of trust. A group of people have always established some sort of society, and for some reason or another, another group of people thinks they can run it better, or deserve it better, and comes and takes it over.

Those people running their everyday lives, more-or-less trusting each other to not kill each other, and some other group comes and kills them. What a disaster. What an abuse of trust. What a horrible, disgusting legacy the human race has left behind in certain epochs. There's no point in accusing any one nationality, or one people group, or even one culture. We're all guilty of it  to greater or lesser degrees, and our ancestors were too. This is just the way human society has worked until now. But things are about to change--for good.

\section{Consensus-Based Epistemology}

For the first time in history, a chain of agreement that goes from zero to hero, and stays at hero, is possible. Imagine a trust matrix. Trust decays over time, and with increasing quantity, and decreasing quality, of human relationships. The more human intermediaries something goes through, the more untrustworthy it potentially becomes.\footnote{plot showing trustworthiness over time of a historical event or action | plot showing trustworthiness over time of a historical event or action on the blockchain | conclusion $\rightarrow$ exponential returns on every zone of human action that depends upon any specific knowledge of any kind (we are going to take off in advancement)}

\subsection{Trust Decay}

How many people does a story need to go through before it stops being reliable? Even a person telling a story can give an unreliable or otherwise incomplete account, even if they don't intend to. This does not even account for the possibility of deception, or intentionally giving an incomplete or otherwise significantly altered story. Trust decay is defined as this: \textit{objects passed through multiple intermediaries\footnote{whether informational or physical or theoretical obects} tend to become obscured as to their history, trustworthiness, origin, and reliability on many different scores and scales.}

This is what makes money laundering possible. Banks and Western Unions didn't set themselves up to become money launderers. It just turns out that it's a lot easier to conceal the source of money if you have one or more intermediaries (people or institutions) to pass it through. The reason it's easy to launder money through a bank is the same reason why time itself can, to a limited extent, launder money and events.  The simple passing of the days keeps things foggy in origin and epistemological certainty is always undermined by a continual process of change.\footnote{I would not argue that this process of change is inherently bad, only that empirically it has this effect.} 

As humans, we persistently look for ways to increase our certainty. We want more signs that our partner loves us, more reasons to believe a loved one is guilty, more people to like us and what we do, more friends, more money, more evidence of all kinds. 

\begin{tikzpicture}[]

\end{tikzpicture}

\section{Practically Infallible History}
This does not mean that it's entirely infallible. Practically infallible history is simply a historical record that's so reliable, comparatively, to anything we've ever known before, that it can basically be considered an infallible record except in a few outlying cases, but infallible according to the group that maintains the consensus mechanism \textit{infinitely}.

\section{New Fields of Philosophy}

\section{The First Truly Quantitative Economics}
\subsection{Crypto Economics}

\subsection{A philosophically guided specification, practiced and implemented in smart contract logic.}

A hundred individuals or groups can put their opinions in here on how Ethereum should be, and it will generate a new specification that adheres to all the Ethereum core principles.
\end{multicols}

\end{document}
