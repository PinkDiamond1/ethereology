\documentclass[a4paper,oneside]{article}
\usepackage{graphicx}
\usepackage[utf8]{inputenc}
\usepackage{ragged2e}
\usepackage[english]{babel}
%\usepackage{draftwatermark}
\usepackage{lipsum}
\usepackage[scaled=.92]{montserrat}
\usepackage{color,calc}
\usepackage[T1]{fontenc}
\usepackage{montserrat}
\usepackage[edges]{forest}
\usepackage{dirtytalk}
\usepackage{listings}
\usepackage{color}
\usepackage{multicol}
\usepackage{setspace}
\usepackage[normalem]{ulem}
\usepackage{geometry}


\title{Ethereology: The Science of Ethereum}
\author{Micah Dameron}
\renewcommand{\baselinestretch}{1.15}


\begin{document}
\maketitle

\begin{abstract}
Ethereum is poised to change the way the world operates in many fundamental ways. It increases certainty, both for individuals and society, by several orders of magnitude. It gives us a single source of truth to rely on that isn't nearly as subject to interpretation as the institutions and models we're used to. This will revolutionize everything the hand of humanity has touched.
\end{abstract}

\tableofcontents

\begin{multicols}{2}

\section{Facilitating Interaction}
The key word is enablement. Ethereum  doesn't necessarily provide humanity with a new way of operating. Rather it forcefully interjects a layer of truth between interactions we already have in our old ways of operating. The goal is not a trustless society, but rather one where trust isn't wasted on institutions that control, temporarily or permanently, offerings to certain life services.

This change will be so in-depth, and so profound, that it will fundamentally and permanently change the primary operation of institutions and individuals across society. So far-reaching and comprehensive will be these changes that the world as we know it today will barely resemble itself in 100 years. We are on the brink of societal and cultural change that is going to make the Industrial Revolution and the Information Revolution look like peanuts. 

Until now humanity has made its living through conflict. Conflicts on the inside, and conflicts on the outside. Conflicts will still exist, of course. But until now our conflicts have held us back significantly, because of their power to destroy. Besides the natural imperative to reproduce and propagate oneself and one's life, there are no forces fundamentally driving the survival of the human race as a whole (let alone that of individual humans.) 

Sure we have our teamwork and our ability to communicate, but that, even under the best of circumstances, these can be so flimsy as to repeatedly fail at the precise moments when the situation calls for them. This is why leadership is always a valued asset. Anyone with the tenacity to bring together the tenuous connections that bind most people into an organized whole seems to be intrinsically worth following. 

None of these positive personality attributes are minimized or neglected by the operation of inherently trustworthy protocols like Ethereum. On the contrary, the actions of the trustworthy protocols allow us to automate commercial and public transactions so that information which is normally garnered trust needs can remain out of commercial transactions where it belongs.

\paragraph{Trust} is a privilege to receive, and one's own right to give or take. Trust fosters close human relationships, and most of the things that make life worth living. Trust is an inherent property in all positive human relationships. Trust is abused when one party gathers information from another and doesn't use that information in their best interests. This is what companies do today.

This is not to say that companies, or corporations are evil per se. They may or may not be. It depends on how they abuse the trust, and use the trust that is given to them. If we step back a few centuries to older societal structures, you'll find that there were kings and nobles, as well as other institutions, that abused people's trust. A thousand years before that, conquesting and "taking" cities was considered a victory. In reality it was an abuse of trust. A group of people have always established some sort of society, and for some reason or another, another group of people thinks they can run it better, or deserve it better, and comes and takes it over.

Those people running their everyday lives, more-or-less trusting each other to not kill each other, and some other group comes and kills them. What a disaster. What an abuse of trust. What a horrible, disgusting legacy the human race has left behind in certain epochs. There's no point in accusing any one nationality, or one people group, or even one culture. We're all guilty of it  to greater or lesser degrees, and our ancestors were too. This is just the way human society has worked until now. But things are about to change--for good.

\section{Consensus-Based Epistemology}

For the first time in history, a chain of agreement that goes from zero to hero, and stays at hero, is possible. Imagine a trust matrix. Trust decays over time, and with increasing quantity, and decreasing quality, of human relationships. The more human intermediaries something goes through, the more untrustworthy it potentially becomes.\footnote{plot showing trustworthiness over time of a historical event or action | plot showing trustworthiness over time of a historical event or action on the blockchain | conclusion $\rightarrow$ exponential returns on every zone of human action that depends upon any specific knowledge of any kind (we are going to take off in advancement)}

\subsection{Decay of Trust}

How many people does a story need to go through before it stops being reliable? Even a person telling a story can give an unreliable or otherwise incomplete account, even if they don't intend to. This does not even account for the possibility of deception, or intentionally giving an incomplete or otherwise significantly altered story. Trust decay is defined as this: \textit{objects passed through multiple intermediaries\footnote{whether informational or physical or theoretical obects} tend to become obscured as to their history, trustworthiness, origin, and reliability on many different scores and scales.}

This is what makes money laundering possible. Banks and Western Unions didn't set themselves up to become money launderers. It just turns out that it's a lot easier to conceal the source of money if you have one or more intermediaries (people or institutions) to pass it through. The reason it's easy to launder money through a bank is the same reason why time itself can, to a limited extent, launder money and events.  The simple passing of the days keeps things foggy in origin and epistemological certainty is always undermined by a continual process of change.\footnote{I would not argue that this process of change is inherently bad, only that empirically it has this effect.} 

As humans, we persistently look for ways to increase our certainty. We want more signs that our partner loves us, more reasons to believe a loved one, more people to like us and what we do, more friends, more money, more evidence of all kinds that something is the way we think it is and that we have control over our respective environments. We hate entertaining doubt, and we hate still more when doubt forces us to entertain it.

Philosophers have dealt with the problem of uncertainty since time immemorial. How can we be sure something is the way we see it, hear it, or perceive it? Many philosophers have leaned on Mathematics\footnote{That some whole can be split into an endless number of component parts, and those parts can be counted, weighed, measured, and multiplied.} for this certainty. Apparently, in spite of its status as an invention of humankind, it continues to provide positive results in every field that it is properly applied.

The amount of certainty invested in a particular fact or set of facts is one of the questions studied by epistemology, the science and study of true knowledge. Epistemologists consider what factors need to be in place for anyone to be certain of one particular thing. Apparently many sources of doubt can creep in to our everyday lives. Many of us doubt what we think or feel. Others doubt their ambitions. And that is to say nothing of the outer world where the causes for doubt are equally expansive and multitudinous.

Everyone learns, for reasons of practicality, that you can't question every aspect of your reality all of the time. You'd never get anywhere. This makes it necessary to create foregone conclusions about our world that may not be necessarily true, but are necessary to meet certain intrinsic needs that we have for order and spontaneity. The truth is, all of us are swimming so deeply in uncertainty that we don't even know we're wet.

We don't know \textit{for certain} that a handful of pennies are all legitimate and that there's no counterfeits in the pile.\footnote{Fortunately for us, counterfeit pennies aren't in all that much demand.} We don't know for certain that we've never possessed a counterfeit \$20.00 bill and used it to pay for some goods. We don't know for certain that the details of what we remember about some particular events are just right. But we do know that we have to take these things for granted most of the time, that is we have to take for granted that our memories are reliable, or else we'd die just thinking about the massive net of possibilities.

This way then, making presumptions and assumptions about sundry affairs of every day life seems like the normal course of things. Most people don't realize that what they take for granted is really far from certain. We might subjectively measure certainty with the following criteria:

\textsc{On a scale of one to ten, how sure are you that $x$ is $y$ at a particular place and a particular time?} 

Where $x$ is an object and $y$ is a condition pertaining in some way to that object. If we were to use this subjective rating system, we would discover that many people, at many places, and in many times would report a different level of certainty about the same object. For example:

\textsc{Were the lights dimmed when you entered the house, or were they on full-strength?}

You would immediately recall the house -- you've seen these lights before on full strength so you know they light up the room very well. As it happens a brighter Halogen light has been installed on your porch by the groundskeeper without your knowledge. Since you stopped to send a final text at the front door, your eyes are dimmed by the brightness of the light. When you step inside, the lights are on full-blast, but they appear dimmed. You tell the court the lights had been dimmed and so there had to have been someone else in the house between when you left and when you came back.

For the court case, this might end up fine, unless the witness depended on your testimony. Maybe they call a mistrial. We trust that the truth will come to light eventually. Now imagine 100 times as many mistakes on 100 other dimensions, and you have something close to the amount of miscommunication that can go on in a single day. Multiply that on a global scale and you have an apparently limitless amount of potential miscommunications between everyone and everything from world leaders to  the wealthy, to the unwealthy, to your own pet dog.

Communication is fundamentally uncertain, and our grip on reality\footnote{not immediate perceptual reality, but long-range time-factored historical and empirical reality} is tenuous. 

If we take this logic and apply it to the realm of contracts, or of human agreements in general, we find still more uncertainty. Nearly the entire legal industry exists to make sure facts are true, on the sole premise that things surrounding a court case are always, or tend always, to be uncertain, as they are surrounding anything. The job of a lawyer is to present the facts of each side in accordance with reality, without losing sight of the subjective factors that might have influenced a particular decision, to put the best foot forward, to bring facts to light and eliminate misunderstandings.

The whole legal industry exists on the premise that by getting input from these certain offices in a certain order, we can reach a certain, definite fact (or at least our closest approximation to one,) and consensus can  be reached on some issue. All facts, whether in a trial or not, are open to this kind of scrutiny. Even the question of the counterfeit penny (an extremely unlikely situation to say the least,) cannot be asserted to be not certain. That is, we cannot say with  definitivity, that a particular stack of pennies ascertained from regular trade with various businesses, contains no counterfeits. But there is a new kind of recorded fact in town, one that doesn't change its truth value over time, can't be fraudulently created or mixed, and answers to no one in particular, but subordinates itself in general to everybody. This is called a \textbf{blockfact}.


\subsection{Blockfacts}
 For the purposes of this demonstration, we will take a blockfact to mean an assertion about reality which is recorded and maintained by the regular operation and use of a blockchain, (decentralized, consensus-based, proof-based blockchains only--nothing centralized or non-deterministic in operation).
 
 The first blockfact to ever exist may be the first Bitcoin transaction. In it, Satoshi Nakamoto (the pseudonymous creator of Bitcoin) sent 10 BTC to Hal Finney, a famous cryptographer. Nearly ten years later, that fact is as true as the day it happened and was recorded. There has been no mishandling of some company records that make us less certain about that fact over time. Instead, we base our certainty on the fact that \textit{such a decision was reached} is something today that miners all across the world re-assert as they mine for the next  Bitcoin block, which inherently must be tied to that first transaction, through a long chain of cryptographic proofs and an enormous amount of computational work over time.
 
 The fact has not faded that this transaction happened. It is still as rich as if it happened yesterday. Our records are clear. We don't risk losing them through mishandling of crucial documents. We don't risk miscalculation. This is a fact, a hard solid fact. This happened. This is a \textit{blockfact}.
 
 \subsection{Practically Infallible History}
 This does not mean that it's entirely infallible. Practically infallible history is simply a historical record that's so reliable, comparatively, to anything we've ever known before, that it can basically be considered an infallible record except in a few outlying cases, but infallible according to the group that maintains the consensus mechanism \textit{infinitely}.
 
\section{Why Ethereology?}

Needless to say

\begin{tikzpicture}[]

\end{tikzpicture}

\section{The First Truly Quantitative Economics}
\subsection{Crypto Economics}

\end{multicols}

\end{document}
